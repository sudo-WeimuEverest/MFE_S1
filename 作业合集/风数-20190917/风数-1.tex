\documentclass{article}
\usepackage{ctex}
\usepackage{amsfonts}
\usepackage{amsmath}
\usepackage{amstext}
\usepackage{amssymb}
\linespread{1.5}
\usepackage[top=1in, bottom=1in, left=1in, right=1in]{
	geometry}

\begin{document}

\begin{flushleft}
	课程名称:《风险管理的数学方法》\\
	提交日期:2019年9月17日\\
	姓名:胡庆涛\\
	学号:1901210003\\ 
\end{flushleft}

\begin{center}
 \LARGE BS公式的推导及解释
\end{center}

\begin{flushleft}
	 $C^{BS}(s,S,r,\sigma,K,T) = SN(d_{1})-Ke^{-r(T-s)}N(d_{2})$\\
	 其中,$d_{1}=\frac{ln(S/K)+(r+\frac{1}{2}\sigma^{2})(T-s)}{\sigma\sqrt{T-s}}$\\
	  $d_{2} = d_{1}-\sigma\sqrt{T-s}$
\end{flushleft}

\section{基本假设}
\begin{enumerate}
	\item 证券无限可分且可被做空
	\item 市场无摩擦
	\item 市场不存在无风险套利机会
	\item 短期无风险利率r为常数,且对所有期限相同
	\item 标的资产的交易连续,且期限内不支付股息
	\item 股票价格服从几何布朗运动(股票的不确定性满足对数正态分布)\\
	$ dS = \mu Sdt + \sigma Sdz $\\
	$\mu$与$\sigma$均为常数\\
	其中,z服从标准布朗运动,满足$ \Delta z = \epsilon\sqrt{\Delta t} $
\end{enumerate}

	$ lnS(T) \sim N [lnS(0)+(\mu-\frac{\sigma ^2}{2})T,\quad \sigma^2T] $
\newpage 
\section{推导思路1:风险中性定价}
\begin{itemize}
	\item 证明对数正态分布的关系式\\
	设$ S $服从对数正态分布,$lnS$标准差为$\omega$,则\\
	\[ E(max(S-K,0)) = E(S)N(d_{1})-KN(d_{2}) \] 
	$ d_{1} = \frac{ln(E(S)/K) + \frac{\omega^2}{2}}{\omega} $\\
	$ d_{2} = \frac{ln(E(S)/K) - \frac{\omega^2}{2}}{\omega} $\\
	证明如下:\\
	定义$ g(S) $ 为S的概率密度函数,则有:
	\[E(max(S-K,0)) = \int_{K}^{\infty}(S-K)g(S)dS\] 
	而$lnS$服从正态分布,且标准差为$ \omega $,均值为$ m = ln(E(S)) - \frac{\omega^2}{2} $\\
	进行标准化,定义$ Q = \frac{lnS - m}{\omega} $\\
	易知,$ h(Q) = \frac{1}{\sqrt{2\pi}} e^\frac{- Q^2}{2}$,进行积分变换可得\\
	\begin{align}
	 E(max(S-K),0) 
	 & = \int_{(lnK-m)/\omega}^{\infty}(e^{Q\omega+m} - K) h(Q)dQ\\
	 \nonumber e^{Q\omega+m}h(Q) 
	 & = \frac{1}{\sqrt{z\pi}}e^{(-Q^2+2Q\omega+2m)}/2 \\
	\nonumber & = \frac{1}{\sqrt{z\pi}}e^{(-(Q - \omega)^2+2Q\omega+2m+\omega^2)}/2\\
	\nonumber & = \frac{e^{m+\omega^2}}{\sqrt{2\pi}}e^{-(Q-\omega)^2/2}\\
	\nonumber & = e^{m+\omega^2/2}h(Q-\omega)
	\end{align}
	故$ (1) $式变为:\\
	\[E(max(S-K),0) 
	=\underbrace{e^{m+\omega^2}\int_{(lnK-m)/\omega}^{\infty}
	h(Q-\omega)dQ}_{\textcircled{1}} 
	-\underbrace{K\int_{(lnK-m)/\omega}^{\infty}h(Q)dQ}_{\textcircled{2}}\]
	定义$ N(x) $表示标准正态分布变量小于$ x $的概率,则\\
	\textcircled{1}式中积分为$ N(\frac{ln(E(S)-K)+\omega^2/2}{\omega})=N(d_{1}) $\\
	\textcircled{2}式积分为$ N(\frac{ln(E(S)-K)-\omega^2/2}{\omega})=N(d_{2}) $\\
	\[ E(max(S-K,0)) = e^{m+\omega^2}N(d_{1})-KN(d_{2}) \] 
	\item 代入欧式看涨期权的情形\\
	时刻$ T$到期,$s$时刻股票价格为$S$,不支付股息,
	期权的执行价格为$K$,无风险利率为$r$,股票价格的波动率为$\sigma $,欧式看涨期权价格为:\\
	\[C^{BS}= e^{-r(T-s)}E(max(S-K,0))\]
	又因在$BS$假定及风险中性世界中,$ E(S)=Se^{-r(T-s)}$,$ lnS $标准差为$ \sigma\sqrt{T} $,代入可得:
	\[C^{BS}(s,S,r,\sigma,K,T) = SN(d_{1})-Ke^{-r(T-s)}N(d_{2})\]
	其中,$d_{1}=\frac{ln(S/K)+(r+\frac{1}{2}\sigma^{2})(T-s)}{\sigma\sqrt{T-s}}$\\
	$d_{2} = d_{1}-\sigma\sqrt{T-s}$
	\item 公式解释\\
	可将$ N(d_{2}) $理解为风险中性世界中期权被行使的概率,$ SN(d_{1})e^{-r(T-s)} $是$ S>K $时为S,
	其他情况为零的变量在风险中性世界中的期望值,故期权在T时刻的期望值等于:
	\[SN(d_{1})e^{-r(T-s)}-KN(d_{2})\]
	贴现到时刻s即为期权的定价公式。
\end{itemize}
\section{推导思路2:伊藤引理}
\begin{itemize}
	\item 推导对数正态分布\\
\begin{align}
	dS = \mu Sdt + \sigma Sdz
\end{align}
	由伊藤引理,对于伊藤过程$ dx = a(t,x)dt + b(t,x)dz$\\
    则对G(x,t),泰勒展开即得:\\
    $dG = (\frac{\partial G}{\partial x}a + \frac{\partial G}{\partial t} +\frac{1}{2}\frac{\partial^2 G}{\partial x^2}b^2)dt
    +\frac{\partial G}{\partial z}bdz	$\\
    对于上式,令$ G = lnS $,各项代入可得\\
\begin{align} 
dlnS = (\frac{\partial G}{\partial x}\mu S + \frac{\partial G}{\partial t} +\frac{1}{2}\frac{\partial^2 G}{\partial x^2}\sigma^2 S^2)dt
+\frac{\partial G}{\partial z}\sigma Sdz 
\end{align}
	整理得,$dlnS = (\mu - \frac{\sigma^2}{2})dt +  \sigma dz$\\
	又$\Delta z = \epsilon\sqrt{\Delta t}$,$ \Delta t= T - s $\\
	则有$ \Delta G = (\mu - \frac{\sigma^2}{2})\Delta t 
	+ \sigma \epsilon\sqrt{\Delta t} $,
	得$\Delta G \sim N [ (\mu - \frac{\sigma^2}{2})\Delta t, \sigma^2 \Delta t]$\\
	而$\Delta lnS = lnS_{T} -lnS_{0}$,
	则$lnS_{T}\sim N[ lnS_{0} + (\mu - \frac{\sigma^2}{2})\Delta t, \sigma^2 \Delta t]$

	\item 对期权价格使用伊藤引理,设期权价格为$C^{BS}$,
	则$C^{BS} = f(S, t)$\\
\begin{align} 
dC^{BS} = (\frac{\partial C^{BS}}{\partial x}\mu S + \frac{\partial C^{BS}}{\partial t} +\frac{1}{2}\frac{\partial^2 C^{BS}}{\partial x^2}\sigma^2 S^2)dt
+\frac{\partial C^{BS}}{\partial z}\sigma Sdz 
\end{align}
\item 利用股票和期权构建投资组合\\
\[\text{卖空1单位期权}
\qquad \text{买入}\frac{\partial C^{BS}}{\partial S}\text{单位的股票} \]
则该组合价值为:$ \Pi = - C^{BS} + \frac{\partial C^{BS}}{\partial S} S$
在时间间隔$ \Delta t $内变化为:\\
\begin{align}
\Delta \Pi =  -\Delta C^{BS} + \frac{\partial C^{BS}}{\partial S} \Delta S
\end{align}
由风险中性,可知$\Delta \Pi = r \Delta t \Pi $
将组合价值上式和(4)式代入(5)式得\\
\begin{align}
\frac{\partial C^{BS}}{\partial t} + r S \frac{\partial C^{BS}}{\partial S}
+ \frac{1}{2} \sigma^2 S^2 \frac{\partial^2 C^{BS}}{\partial S^2} 
= r C^{BS}
\end{align}
上式(6)即为BS公式的微分方程\\
该方程边界条件为:$ t = T \text{时,} C^{BS} = max(S-K,0)$\\
求解可得:$C^{BS}(s,S,r,\sigma,K,T) = SN(d_{1})-Ke^{-r(T-s)}N(d_{2})$\\
其中, $d_{1}=\frac{ln(S/K)+
		(r+\frac{1}{2}\sigma^{2})(T-s)}{\sigma\sqrt{T-s}}$\\
	$d_{2} = d_{1}-\sigma\sqrt{T-s}$

\end{itemize}

\section{希腊值的推导及解释}
上述公式中s以年为单位计算,现将其进行转换,记作$ t\Delta $,t以日历天计数,期权的在t天的价值为:\\
\[V_{t}= C^{BS}(t\Delta, S_{t, r_{t}, \sigma_{t}, K, T})\]
风险因子改变记为:\
\[X_{t+1}= (lnS_{t+1}-lnS_{t}, r_{t+1}-r_{t}, \sigma_{t+1}-\sigma_{t})\]
故损失的一阶近似为:\
\[L^{\Delta}_{t+1}= -(C^{BS}_{s}\Delta + C^{BS}_{S}S_{t}X_{t+1,1}
+ C^{BS}_{r}X_{t+1,2}+ C^{BS}_{\sigma}X_{t+1,3})\]
欧式看涨期权价格为:
\[C^{BS}(s,S,r,\sigma,K,T) = SN(d_{1})-Ke^{-r(T-s)}N(d_{2})\]
其中,$ d_{1}=\frac{ln(S/K)+
	(r+\frac{1}{2}\sigma^{2})(T-s)}{\sigma\sqrt{T-s}}$\\
$d_{2} = d_{1}-\sigma\sqrt{T-s}$\\
$N(x)$表示标准正态分布变量小于$ x $的概率,故$ N^{'}(x)= \frac{1}{\sqrt{2\pi}} e^{-\frac{x^2}{2}}$
\begin{align}
\nonumber N^{'}(d_{1}) & = N^{'}(d_{2}+\sigma\sqrt{T-s})\\
\nonumber  & = \frac{1}{\sqrt{2\pi}}exp[-\frac{d_{2}^2}{2}-\sigma d_{2}\sqrt{T-s}
-\frac{1}{2}\sigma^2](T-s)\\
\nonumber  & =N^{'}(d_{2}exp[-\sigma d_{2}\sqrt{T-s}-\frac{1}{2}\sigma ^{2}(T-s)])
\end{align}
故有,
\[SN^{'}(d_{1}) = Ke^{-r(T-s)}N^{'}(d_{2})\]
\begin{itemize}
	\item 期权的Delta\\
	期权的Delta定义为期权价格变动与标的资产价格变动的比率\\
	故对于欧式看涨期权\\
	\begin{align}
	\nonumber \Delta & = \frac{\partial C^{BS}}{\partial S}\\
	\nonumber  & = N(d_{1}) 
	\end{align}
	\item 期权的Gamma\\
	期权的Gamma表示期权Delta随标的资产价格变化的比率,即期权价格关于标的资产价格的二阶导,度量期权价格与标的资产价格关系的曲率\\
	\begin{align}
	\nonumber \Gamma & = \frac{\partial^2 C^{BS}}{\partial S^2}\\
	\nonumber  & = \frac{N^{'}(d_{1})}{S\sigma\sqrt{T-s}} 
	\end{align}
	\item  期权的Theta\\
	期权的Theta定义为期权价格变动随时间变化的比率\\
	故对于欧式看涨期权\\
	\begin{align}
	\nonumber \Theta & = \frac{\partial C^{BS}}{\partial s}\\
	\nonumber  & = SN^{'}(d_{1}\frac{\partial d_{1}}{\partial s}) -
	rKe^{-r(T-s)N(d_{2})}-Ke^{-r(T-s)}N^{'}(d_{2})\frac{\partial d_{2}}{\partial s}\\
	\nonumber  & =-rKe^{-r(T-s)}N(d_{2}) + SN^{'}(d_{1})(\frac{\partial d_{1}}{\partial s}-\frac{\partial d_{2}}{\partial s})\\
	\nonumber  & = -rKe^{-r(T-s)}N(d_{2}) + SN^{'}(d_{1})\frac{\partial }{\partial s}(\sigma\sqrt{T-s})\\
	\nonumber  & = -rKe^{-r(T-s)}N(d_{2}) - SN^{'}(d_{1})\frac{\sigma}{2\sqrt{T-s}}
	\end{align} 
	该公式s以年为单位,可进行转换计算天的Theta。
	\item Delta、Theta与Gamma的关系\\
	上文推导BS公式的微分方程时得到公式(6)如下:\\
	\[\frac{\partial C_{BS}}{\partial t} + r S \frac{\partial C_{BS}}{\partial S}
	+ \frac{1}{2} \sigma^2 S^2 \frac{\partial^2 C_{BS}}{\partial S^2} 
	= r C^{BS}\]
	分别代入三个希腊值后可得\\
	\[\Theta + rS \Delta + \frac{1}{2}\sigma^2S^2\Gamma = r C^{BS}\]
	\item 期权的Rho\\
	期权的Rho定义为期权价格变动随利率变化的比率\\
	故对于欧式看涨期权\\
	\begin{align}
	\nonumber \varrho & = \frac{\partial C^{BS}}{\partial r}\\
	\nonumber  & = K(T-s)e^{-r(T-s)}N(d_{2}) 
	\end{align}
	\item 期权的Vega\\
	期权的Vega定义为期权价格变动随资产波动率变化的比率\\
	故对于欧式看涨期权\\
	\begin{align}
	\nonumber \mathcal{V} & = \frac{\partial C^{BS}}{\partial \sigma}\\
	\nonumber  & = SN^{'}(d_{1})\frac{\sqrt{T-s}}{2} + Ke^{-r(T-s)}N^{'}(d_{2}) \frac{\sqrt{T-s}}{2}
	\nonumber  & =S\sqrt{T-s}N^{'}(d_{1})
	\end{align}
\end{itemize}
\end{document}